\documentclass{hyexc}

\subimport{../}{title}

\begin{document}
\maketitle

\valiteksti{\noindent Tällä kurssilla kerrataan lukiomatematiikan osa-alueita suurin piirtein siinä laajudessa, kun niitä kursseilla 'Johdatus yliopistomatematiikkaan' ja 'Analyysiä reaaliluvuilla' tarvitaan. 

Kurssi suoritetaan tekemällä tehtäviä, eikä kurssikoetta järjestetä.  Tehtävät tulee luonnollisesti tehdä huolella ja palauttaa ajankohtaan mennessä. Kurssiin kuuluu olennaisesti myös verkosta löytyvät Stack-tehtävät, jotka tulee suorittaa ennen palautuspäivämäärää. Kurssi on mahdollista suorittaa yhden tai kahden opintopisteen laajuisena.

Tehtäviä kannattaa miettiä yhdessä toisten kurssilaisten kanssa, sillä valmiita vastauksia ei ole saatavilla. Tehtäviin saa myös apua viikottaisissa tapaamisissa, joissa käyminen on suositeltavaa, joskaan ei pakollista.  %Tehtäviin saa apua viikoittaisissa tapaamisissa, mutta koulusta tuttuja `vastauksia' ei ole saatavilla. Jokainen vaihe on siis syytä tehdä ja tarkistaa huolella. Ratkaisuja kannattaa myös verrata toisten kurssilaisten kanssa ja selvittää etukäteen mahdollisia ongelmakohtia.
}
\tableofcontents
\part{Lausekkeiden käsittely ja tuntemus (1 op)}
\valiteksti{Tässä osassa harjoitellaan matematiikan mekaanisempaa puolta. Tarkoitus on kehittää laskurutiinia ja saada varmuutta mm. lausekkeiden ja yhtälöiden käsittelyyn. Myös funktioihin liittyvät peruskäsitteet tulevat kerratuksi, samoin logaritmin tärkeimmät ominaisuudet. Viimeiseen osioon on koottu sekalaisia tehtäviä mm. analyyttisen geometrian asioista.
}
\subimport{.}{lala.tex}
\vfill
\subimport{.}{fuyh.tex}
\vfill
\subimport{.}{pola.tex}
\vfill
\subimport{.}{sela1.tex}
\vfill

\end{document}
