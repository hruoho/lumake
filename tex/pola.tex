
\begin{tehtava}
Sievennä
\begin{kohdat}
\item \(\left(x^3\right)^2 \cdot x\)
\item \(\left(\left( y^3 \right)^3 \right)^2\)
\item \(\left( x^{\frac{2}{6}} \right)^3\)
\end{kohdat}
\end{tehtava}

\begin{tehtava}
Sievennä
\begin{kohdat}
\item \(x^{3^2} \cdot x\)
\item \(y^{x^2} \cdot y^{x^2}\)
\item \(\displaystyle \frac{\left(2a^2\right)^4}{2a^5}\)
\end{kohdat}
\end{tehtava}

\begin{tehtava}
Laske ja sievennä
\begin{kohdat}
\item \(2^n\cdot 5^n\)
\item \(6a^2\cdot 5a^3\)
\item \((6a^6):(2a^2)\)
\item \((3ab^2)^3\)
\end{kohdat}
\end{tehtava}

\begin{tehtava}
Muuta yksinkertaisempaan muotoon soveltamalla potenssin laskusääntöjä
\begin{kohdat}
\item \(a^3b^{-2}a^{-1}b^{-3}\)
\item \(\displaystyle \frac{x^2y^4}{x^3y^2}\)
\item \(\displaystyle \left(\frac{a^2x}{b^4y}\right)^2\)
\item \(\displaystyle \left(\frac{p^2q}{r^{-1}s}\right)^{-3}\)
\end{kohdat}
\end{tehtava}

\begin{tehtava}
Sievennä\[\left( \sqrt{ \frac{x-2}{x+3} } \right)^4 \cdot \frac{1}{x-2}\]
\end{tehtava}

\begin{tehtava}
Sievennä
\begin{kohdat}
\item \(\displaystyle \sqrt{ \frac{x^4}{8} }\)
\item \(\displaystyle \sqrt{ \frac{(x-1)^2}{4} }\)
\end{kohdat}
\end{tehtava}

\begin{tehtava}
Sievennä
\begin{kohdat}
\item \(\sqrt{x^2}\)
\item \(\sqrt{t^2-2t + 1}\)
\item \(\displaystyle \frac{\sqrt{a^2b}}{\sqrt{a}\sqrt{b^2}}\)
\item \(\sqrt{12s}\sqrt{3s^3}\)
\end{kohdat}
\end{tehtava}

\begin{tehtava}
Kirjoita murtopotenssimerkintää käyttäen
\begin{kohdat}
\item \(\sqrt{x^2}\)
\item \(\displaystyle \left( \sqrt{x} \right)^2\)
\end{kohdat}
\end{tehtava}

\begin{tehtava}
Muuta murtopotenssimuotoon.
\begin{kohdat}
\item \(\displaystyle \sqrt{\frac{1}{a^5}}\)
\item \(\sqrt{\sqrt{2}}\)
\end{kohdat}
\end{tehtava}

\begin{tehtava}
Sievennä
\begin{kohdat}
\item \(\left( \sqrt{x^2} \right)^6\)
\item \(\left(x^{-2} \cdot y^2\right)^2\)
\end{kohdat}
\end{tehtava}

\begin{tehtava}
Sievennä
\begin{kohdat}
\item \(3^x \cdot 3^{2x}\)
\item \(a^{x+y} \cdot a^x\)
\end{kohdat}
\end{tehtava}

\begin{tehtava}
Sievennä
\begin{kohdat}
\item \(2^x \cdot 2^{3x}\)
\item \(a^{2x +1} \cdot a^{-(x+1)}\)
\end{kohdat}
\end{tehtava}

\begin{tehtava}
Sievennä
\begin{kohdat}
\item \(\displaystyle \frac{1}{x^2} \cdot x^{-3}\)
\item \(\displaystyle \frac{2y}{3} \cdot y^{-2}\)
\end{kohdat}
\end{tehtava}

\begin{tehtava}
Millä luvuilla $x$ pätee
\begin{kohdat}
\item \(x^2 = 16\)
\item \(x^3 = 8\)
\item \(x^4 = 16\) ?
\end{kohdat}
\end{tehtava}

\begin{tehtava}
Olkoon $f(x) = x^2$ ja $g(x)= \sqrt{x}$. Määritä
\begin{kohdat}
\item \(g(f(2))\)
\item \(g(f(x)).\)
\end{kohdat}
\end{tehtava}

\begin{tehtava}
Millä luvulla $x$ pätee $5^{x} = 125$? Entä $5^{2x-1} = 125$?
\end{tehtava}

\begin{tehtava}
Olkoon $f(x) = 4^x$. Millä muuttujan $x$ arvolla pätee $f(x)=64$?
\end{tehtava}

\begin{tehtava}
Laske
\begin{kohdat}
\item \(\log_4 64\)
\item \(\log_{10} (10^6)\)
\item \(\log_7 1\)
\item \(\log_9 9.\)
\end{kohdat}
\end{tehtava}

\begin{tehtava}
Olkoon $g(x) = \log_2 x$. Mikä on funktion $g$ arvo pisteessä $16$?
\end{tehtava}

\begin{tehtava}
Millä luvulla $x$ pätee $\log_5 (2x-1) = 2$?
\end{tehtava}

\begin{tehtava}
Sievennä\[\frac{1}{x} \cdot \log_2 (8^x).\]
\end{tehtava}

\begin{tehtava}
Olkoon $f(x) = \log_2 x$. Laske $f(\frac{1}{8})$, $f(\frac{1}{4})$, $f(1)$, $f(4)$ ja $f(8)$. Hahmottele funktion kuvaaja.
\end{tehtava}

\begin{tehtava}
Päättele logaritmin määritelmän nojalla (ilman laskinta), mitä ovat\[\log_5 125, \qquad \log_3\sqrt{3}, \qquad \log_2\frac{1}{4}, \qquad \log_{10}10^6, \qquad \log_2 8^x.\]Tarkista lopuksi tuloksesi laskemalla logaritmit laskimella (viimeistä lukuunottamatta). Tarvitset tähän kantaluvun muunnoskaavaa\[\log_a x=\frac{\ln x}{\ln a}.\]
\end{tehtava}

\begin{tehtava}
Ratkaise yhtälöt\[e^{2x}=1200 \qquad \text{ja} \qquad \ln(5x-1)=2.\]
\end{tehtava}
